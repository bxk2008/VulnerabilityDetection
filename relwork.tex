
\section{Related work}
\label{sec:relwork}

In \cite{Ma2017VuRLE},the authors propose a new tool called VuRLE for automatic detection and repair of vulnerabilities.  From examples of vulnerable codes and their corresponding repaired codes ,the edit blocks are firstly extracted by performing Abstract Syntax Tree (AST) diff ,which then are divided into edit groups by a DBSCAN \cite{Ester1996A}-inspired clustering algorithm , and finally generate the repair templates corresponding to each edit group which contain edit pattern and a context pattern where uses the context patterns to detect vulnerabilities and customizes the corresponding edit patterns to repair them.Experiment with vulnerabilities from real-world applications, showing that the VuRLE outperforms the other  tool called LASE, which results a large improvement on accuracy of detection and the number of repaired vulnerabilities.\\
In \cite{Chernis:2018:MLM:3180445.3180453},for catching a large percentage of bugs ,the text features including simple and complex features are extracted from functions in C source code and then the Naive Bayes was used as the default classifier to analyze them.Experiment shows that simple features performed unexpectedly better compared to the complex features.\\
In \cite{DBLP:conf/ndss/LiZXO0WDZ18}, the authors present a deep learning-based system for vulnerability detection, called VulDeePecker.The features are obtained by first extracting code gadgets from the buggy programs, and then transformed into vector representations,where the code gadgets are fragments of codes that are semantically related to each other.The learning algorithm is based on the Long Short-Term Memory (LSTM).Experimental evaluation on the public available vulnerability dataset demonstrates that VulDeePecker outperforms the other state-of-the-art code similarity-based vulnerability detection systems in terms of both accuracy and efficiency.\\
In \cite{Harer2018Automated},propose a model combins features learned by deep models with tree-based models which is applied directly to source code can output over fifty different categories of weakness in Common Weakness Enumeration (CWE)\cite{MITRE}.\\
In \cite{Russell2018Automated},the authors develop a vulnerability detection tool based on deep feature representation learning that directly interprets lexed source code.The source codes are firstly tansformed to tokens which are then embedding to vectors,and the convolutional neural networks (CNNs) and recurrent neural networks(RNNs) are used for feature extraction to realize function-level source vulnerability classification.Experiment with real software packages and the NIST SATE IV benchmark dataset,showing that the model proposed with deep feature representation
learning on source code results high accuracy.\\
In \cite{Zhen2018SySeVR},the authors propose the first systematic framework for using deep learning to detect vulnerabilities which dubbed Syntax-based, Semantics-based, and Vector Representations (SySeVR) that can accommodate syntax and semantic information pertinent to vulnerabilities.The source codes are successively represented by Syntax-based, Semantics-based, and Vector,finally,train the BLSTM with  Vector Representations. Experiments with 4 software products shows that framework can detect 15 vulnerabilities that are not reported in the National Vulnerability Database.\\
In \cite{Comparative-Study},the authors firstly collect two datasets from the programs involving 126 types of vulnerabilities with with data dependency and control dependency  based on an extended open source parser Joern \cite{Joern},and then evaluate the quantitative impact of different factors on the effectiveness of vulnerability detection.Expreiment shows a better result than VulDeePecker.
