
%\documentclass[12pt]{report}

\documentclass[runningheads]{llncs}

\usepackage{graphicx}
\usepackage{color}
\usepackage{listings}

%\input{commands}

%\pagestyle{empty} % no page numbering

\lstset{
language=C,
basicstyle=\small\sffamily,
numbers=left,
numberstyle=\tiny,
frame=tb,
columns=fullflexible,
showstringspaces=false
}


\begin{document}
\title{Vulnerability detection using neural networks with random weights }
%\date{3rd, March, 2016}
\author{ Jingjing Zhang\inst{1} \and Qiang Wang\inst{2}\thanks{Corresponding  author} \and Lin Yang\inst{2}}

\authorrunning{J. Zhang et al.}
\institute{Army Engineering University of PLA, Nanjing, China \\
 Chinese Academy of Military Science, Beijing, China }


\maketitle
%\large

%\let\thefootnote\relax\footnotetext{This work is partially funded by .}
%\footnotetext[0]{This work is partially funded by.}

\begin{abstract}
%

%
\keywords{Security protocol \and 5G Authentication  \and Formal verification \and ProVerif}
\end{abstract}


%{\color{red} TODO: formalize the modeling and verification approach, and then carry out the 5G protocol case study.}
%The formal analysis of security protocols is supported by a vast space of modelling formalisms and tools.
%The variety of incompatible formats and tools however poses a significant challenge to practical adoption as well as continued research.
%In this paper, we propose a  unified modeling and verification framework for security protocols.
%The modeling framework is based on UML and has been designed for ease of verification without sacrificing readability.
%Several existing tools are integrated as the verification backend, such as ProVerif, Scyther, Tarmarin.
%A transformer has also been implemented to automatically transform the UML model to the one supported by a specific verifier.
%The ultimate purpose of this work is to simplify tool development, encourage research cooperation, and pave the
%way towards a future competition in security protocol verification.


\section{Introduction}
\label{sec:intro}









\section{Verification results}
\label{sec:result}




\section{Related work}
\label{sec:relwork}

In \cite{DBLP:conf/ndss/LiZXO0WDZ18}, the authors present a deep learning-based system for vulnerability detection, called VulDeePecker.
 The features are obtained by first extracting code gadgets from the buggy programs, and then transformed into vector representations,
 where the code gadgets are fragments of codes that are semantically related to each other.
 The learning algorithm is based on the Long Short-Term Memory (LSTM).
 Experimental evaluation on the public available vulnerability dataset demonstrates that VulDeePecker outperforms the other state-of-the-art 
 code similarity-based vulnerability detection systems in terms of both accuracy and efficiency.


\section{Conclusion}
\label{sec:conclusion}




\bibliographystyle{splncs04}
\bibliography{main}

%\appendix
%\section{Simplied 5G EAP-TLS authentication protocol}
%\label{appendix:simpro}



\end{document}

